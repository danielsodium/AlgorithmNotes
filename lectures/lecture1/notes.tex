\documentclass[oneside]{book}


\usepackage{amsmath, amsthm, amssymb, amsfonts}
\usepackage{thmtools}
\usepackage{graphicx}
\usepackage{setspace}
\usepackage{geometry}
\usepackage{float}
\usepackage{hyperref}
\usepackage[utf8]{inputenc}
\usepackage[english]{babel}
\usepackage{framed}
\usepackage[dvipsnames]{xcolor}
\usepackage{environ}
\usepackage{tcolorbox}
\usepackage{listings}
\usepackage{parskip}
\tcbuselibrary{theorems,skins,breakable}
\usepackage{xspace}
\newcommand{\opt}{{\scshape opt}\xspace}

\setstretch{1.2}
\geometry{
    textheight=9in,
    textwidth=5.5in,
    top=1in,
    headheight=12pt,
    headsep=25pt,
    footskip=30pt
}

% Variables
\def\notetitle{MATH 101}
\def\noteauthor{
    \textbf{Professor} \\ 
    {\LaTeX} by Joe\\
    University}
\def\notedate{Semester}
% The theorem system and user-defined commands

% The following boxes are provided:
%   Definition:     \defn 
%   Theorem:        \thm 
%   Lemma:          \lem
%   Corollary:      \cor
%   Proposition:    \prop   
%   Claim:          \clm
%   Fact:           \fact
%   Proof:          \pf
%   Example:        \ex
%   Remark:         \rmk (sentence), \rmkb (block)
% Suffix
%   r:              Allow Theorem/Definition to be referenced, e.g. thmr
%   p:              Add a short proof block for Lemma, Corollary, Proposition or Claim, e.g. lemp
%                   For theorems, use \pf for proof blocks

% Definition
\newtcbtheorem[number within=section]{mydefinition}{Definition}
{
    enhanced,
    frame hidden,
    titlerule=0mm,
    toptitle=1mm,
    bottomtitle=1mm,
    fonttitle=\bfseries\large,
    coltitle=black,
    colbacktitle=green!20!white,
    colback=green!10!white,
}{defn}

\NewDocumentCommand{\defn}{m+m}{
    \begin{mydefinition*}{#1}{}
        #2
    \end{mydefinition*}
}

\newtcbtheorem[number within=section]{problem}{Problem}
{
    enhanced,
    frame hidden,
    titlerule=0mm,
    toptitle=1mm,
    bottomtitle=1mm,
    fonttitle=\bfseries\large,
    coltitle=black,
    colbacktitle=red!20!white,
    colback=red!10!white,
}{defn}

\NewDocumentCommand{\prb}{m+m}{
    \begin{problem*}{#1}{}
        #2
    \end{problem*}
}


% Theorem
\newtcbtheorem[use counter from=mydefinition]{mytheorem}{Theorem}
{
    enhanced,
    frame hidden,
    titlerule=0mm,
    toptitle=1mm,
    bottomtitle=1mm,
    fonttitle=\bfseries\large,
    coltitle=black,
    colbacktitle=cyan!20!white,
    colback=cyan!10!white,
}{thm}

\NewDocumentCommand{\thm}{m+m}{
    \begin{mytheorem}{#1}{}
        #2
    \end{mytheorem}
}

\NewDocumentCommand{\thmr}{mm+m}{
    \begin{mytheorem}{#1}{#2}
        #3
    \end{mytheorem}
}

% Lemma
\newtcbtheorem[use counter from=mydefinition]{mylemma}{Lemma}
{
    enhanced,
    frame hidden,
    titlerule=0mm,
    toptitle=1mm,
    bottomtitle=1mm,
    fonttitle=\bfseries\large,
    coltitle=black,
    colbacktitle=violet!20!white,
    colback=violet!10!white,
}{lem}

\NewDocumentCommand{\lem}{m+m}{
    \begin{mylemma*}{#1}{}
        #2
    \end{mylemma*}
}

\newenvironment{lempf}{
	{\noindent{\it \textbf{Proof for Lemma}}}
	\tcolorbox[blanker,breakable,left=5mm,parbox=false,
    before upper={\parindent15pt},
    after skip=10pt,
	borderline west={1mm}{0pt}{violet!20!white}]
}{
    \textcolor{violet!20!white}{\hbox{}\nobreak\hfill$\blacksquare$} 
    \endtcolorbox
}

\NewDocumentCommand{\lemp}{m+m+m}{
    \begin{mylemma}{#1}{}
        #2
    \end{mylemma}

    \begin{lempf}
        #3
    \end{lempf}
}

% Corollary
\newtcbtheorem[use counter from=mydefinition]{mycorollary}{Corollary}
{
    enhanced,
    frame hidden,
    titlerule=0mm,
    toptitle=1mm,
    bottomtitle=1mm,
    fonttitle=\bfseries\large,
    coltitle=black,
    colbacktitle=orange!20!white,
    colback=orange!10!white,
}{cor}

\NewDocumentCommand{\cor}{+m}{
    \begin{mycorollary}{}{}
        #1
    \end{mycorollary}
}

\newenvironment{corpf}{
	{\noindent{\it \textbf{Proof for Corollary.}}}
	\tcolorbox[blanker,breakable,left=5mm,parbox=false,
    before upper={\parindent15pt},
    after skip=10pt,
	borderline west={1mm}{0pt}{orange!20!white}]
}{
    \textcolor{orange!20!white}{\hbox{}\nobreak\hfill$\blacksquare$} 
    \endtcolorbox
}

\NewDocumentCommand{\corp}{m+m+m}{
    \begin{mycorollary}{}{}
        #1
    \end{mycorollary}

    \begin{corpf}
        #2
    \end{corpf}
}

% Proposition
\newtcbtheorem[use counter from=mydefinition]{myproposition}{Proposition}
{
    enhanced,
    frame hidden,
    titlerule=0mm,
    toptitle=1mm,
    bottomtitle=1mm,
    fonttitle=\bfseries\large,
    coltitle=black,
    colbacktitle=yellow!30!white,
    colback=yellow!20!white,
}{prop}

\NewDocumentCommand{\prop}{+m}{
    \begin{myproposition}{}{}
        #1
    \end{myproposition}
}

\newenvironment{proppf}{
	{\noindent{\it \textbf{Proof for Proposition.}}}
	\tcolorbox[blanker,breakable,left=5mm,parbox=false,
    before upper={\parindent15pt},
    after skip=10pt,
	borderline west={1mm}{0pt}{yellow!30!white}]
}{
    \textcolor{yellow!30!white}{\hbox{}\nobreak\hfill$\blacksquare$} 
    \endtcolorbox
}

\NewDocumentCommand{\propp}{+m+m}{
    \begin{myproposition}{}{}
        #1
    \end{myproposition}

    \begin{proppf}
        #2
    \end{proppf}
}

% Claim
\newtcbtheorem[use counter from=mydefinition]{myclaim}{Claim}
{
    enhanced,
    frame hidden,
    titlerule=0mm,
    toptitle=1mm,
    bottomtitle=1mm,
    fonttitle=\bfseries\large,
    coltitle=black,
    colbacktitle=pink!30!white,
    colback=pink!20!white,
}{clm}


\NewDocumentCommand{\clm}{m+m}{
    \begin{myclaim*}{#1}{}
        #2
    \end{myclaim*}
}

\newenvironment{clmpf}{
	{\noindent{\it \textbf{Proof for Claim.}}}
	\tcolorbox[blanker,breakable,left=5mm,parbox=false,
    before upper={\parindent15pt},
    after skip=10pt,
	borderline west={1mm}{0pt}{pink!30!white}]
}{
    \textcolor{pink!30!white}{\hbox{}\nobreak\hfill$\blacksquare$} 
    \endtcolorbox
}

\NewDocumentCommand{\clmp}{m+m+m}{
    \begin{myclaim*}{#1}{}
        #2
    \end{myclaim*}

    \begin{clmpf}
        #3
    \end{clmpf}
}

% Fact
\newtcbtheorem[use counter from=mydefinition]{myfact}{Fact}
{
    enhanced,
    frame hidden,
    titlerule=0mm,
    toptitle=1mm,
    bottomtitle=1mm,
    fonttitle=\bfseries\large,
    coltitle=black,
    colbacktitle=purple!20!white,
    colback=purple!10!white,
}{fact}

\NewDocumentCommand{\fact}{+m}{
    \begin{myfact}{}{}
        #1
    \end{myfact}
}


% Proof
\NewDocumentCommand{\pf}{+m}{
    \begin{proof}
        [\noindent\textbf{Proof.}]
        #1
    \end{proof}
}

% Example
\newenvironment{example}{%
    \par
    \vspace{5pt}
	\begin{minipage}{\textwidth}
		\noindent\textbf{Example.}
		\tcolorbox[blanker,breakable,left=5mm,parbox=false,
	    before upper={\parindent15pt},
	    after skip=10pt,
		borderline west={1mm}{0pt}{cyan!10!white}]
}{%
		\endtcolorbox
	\end{minipage}
    \vspace{5pt}
}

\NewDocumentCommand{\ex}{+m}{
    \begin{example}
        #1
    \end{example}
}


% Remark
\NewDocumentCommand{\rmk}{+m}{
    {\it \color{blue!50!white}#1}
}

\newenvironment{remark}{
    \par
    \vspace{5pt}
    \begin{minipage}{\textwidth}
        {\par\noindent{\textbf{Remark.}}}
        \tcolorbox[blanker,breakable,left=5mm,
        before skip=10pt,after skip=10pt,
        borderline west={1mm}{0pt}{cyan!10!white}]
}{
        \endtcolorbox
    \end{minipage}
    \vspace{5pt}
}

\NewDocumentCommand{\rmkb}{+m}{
    \begin{remark}
        #1
    \end{remark}
}






\begin{document}
\chapter{Greedy Algorithms and Exchange Argument}
\section{Greedy Algorithms}
\defn{Greedy Algorithm} {
    Choosing the best option at the current step. There is no real "formal" definition of a greedy algorithm.
}
\ex {
    For the Travelling Salesperson Problem, we would go to the clostest location at every step. This would be suboptimal (except in special cases) but greedy.
}
\prb{Scheduling} {
    Given: $n$ jobs. Job $i$ has runtime $t_i$ and deadline $d_i$ \\
    Goal: Form a schedule to minimize the maximum lateness $L = $ max $l_i$\\
    This problem can be thought of as assembly line jobs or conference meetings, one cannot start without the last job ending.
}
\textbf{After you define a problem, don't just try to solve it immediately. Instead think and play around with it.} This will allow you to find edgecases and also counterexamples.
\\\\
We can define the finish time of the job in our schedule to be $f(i)$ and $l(i)$ to be $f(i) - d_i$.\\
 \ex {
    We have two possible schedules. One causes all jobs to be late by 99 minutes, while the other causes all jobs to be on time except for one that is 100 minutes late.
}
Although we may think that the second is a better option due to the fact that most jobs are in time, our objective in the algorithm is only to minimize the worst lateness (Egalitarian), which will it cause it to choose the first option.
\\\\
Possible Greedy Algorithms
\begin{enumerate}
    \item Shortest job first (sort by $t_i$ )
    \item Earliest job first (sort by $t_i$ )
    \item Smallest slack time first (sort by $d_i - t_i$)
\end{enumerate}
\textbf{It is easy to come up with greedy algorithms, but it is hard to tell/prove if one is correct.}
\\ 
We can create contradictions for two of the algorithms.
\begin{enumerate}
    \item Two jobs, one with $t = 1$ and $d = 3$, and another with $t = 2$ and $d = 2$.
    \item Correct algorithm.
    \item Two jobs, one with $ t = 100$ and $d = 100$, and another with $t = 1$ and $d = 2$.
\end{enumerate}

\section{Proving the Algorithm}

First, we rename jobs in order of deadline so that \[
d_1 \leq d_2 \leq d_3 \leq ... \leq d_n
.\] 

\clm {Maximum lateness stays the same no matter how we order jobs with the same deadline} {
    Take an arbituary number of jobs with the same deadline $d$. If maximum lateness happens before or after $d$, the max lateness will not hange. If maximum lateness is equal to $d$, the rightmost task will be the max lateness and therefore will not change no matter the order due to the fact that the times of the jobs that are $d$ deadline being constant. 
}

We can now start to prove using the exchange argument.

\defn{Exchange Argument} {
   Consider an optimal solution which is most similar to the Greedy Solution. We can use proof by cases to compare OPT (the optimal solution) and the Greedy Solution.\\
   \begin{enumerate}
       \item OPT = Greedy\\ In this case, by reiteration the Greedy Solution is the optimal solution.
        \item We try to change OPT to be more like the Greedy solution without making it worse (but still keep it optimal).\\ This is a contradiction because we cannot make the OPT any more similar to the greedy argument.
   \end{enumerate}
}

Now we can use this on the scheduling problem.

\defn{Inversion} {
    Jobs such that $i$ is before $j$ in the schedule but $d_j < d_i$ (basically out of order from Greedy).
}

A Greedy schedule would have zero inversions (no differences from itself). How similar to the Greedy schedule is how few inversion exist.

\lem {} {
   If an inversion exists, there exists a pair of adjacent jobs which are inverted. 
    \pf {
        We can prove this by contradiction. We have jobs $i$ and $j$ that are inverted. We have $k_1, k_2, ... k_n$ jobs between the two. We can then say that
    \[
        d_j < d_i < d_{k_1} < d_{k_n}
    \] 
    Because of this, we can say that $j$ and $k_n$ are inversions, which are adjacent. 
    }
}
Now we can apply the Exchange argument. Consider an optimal schedule which has fewest inversions.
\begin{enumerate}
    \item OPT has 0 inversions
    \item We can decrease the number of inversions in OPT to make it more similar to greedy.
\end{enumerate}
We can then make a claim for case 2 that

\clm {} {
    Swapping adjacent inverted jobs in OPT does not increase max lateness.
}

By this claim, case 2 is a contradiction because the optimal schedule we have has the fewest inversions. Therefore, if we prove this claim, only case 1 could be possible and OPT = Greedy.

\pf {
    We have that $d_i < d_j$. We start with $j$ before $i$ with $j$ starting at time $F$. We then swap $i$ and $j$.
$$ l_i = F + t_j + t_i - d_i $$
$$ l_j = F + t_j - d_j $$
which turns into
$$ l_j' = F + t_i + t_j - d_j $$
$$ l_i' = F + t_i - d_i $$
$$ d_i < d_j \rightarrow l_j' < l_i' \leq L_{max} $$

}

Therefore, the greedy algorithm by earliest deadline is the optimal solution.
\end{document}
